%*******************************************************
% German Abstract
%*******************************************************

\cleardoublepage%

\begingroup
\let\clearpage\relax
\let\cleardoublepage\relax
\let\cleardoublepage\relax

\begin{otherlanguage}{ngerman}
\pdfbookmark[1]{Zusammenfassung}{Zusammenfassung}
\chapter*{Zusammenfassung}

User interfaces (UI) sind wesentliche Bestandteile komplexer moderner Computerplattformen, da sie vorschreiben, wie Menschen Daten in diese Systeme eingeben und wie sie Ausgaben von den Plattformen interpretieren. Viele sicherheitskritische Cyber-Physical-Remote-Systeme wie industrielle PLC (in der Fertigung, in Kraftwerken usw.) und medizinische Implantate sind \"uber umfangreiche Benutzeroberfl\"achen \"uber Browser oder dedizierte Anwendungen zug\"anglich, die auf handels\"ublichen Systemen oder Hosts ausgef\"uhrt werden. In \"ahnlicher Weise sind E-Banking, E-Voting, soziale Netzwerke und viele andere Remote-Anwendungen und -Dienste entscheidend von UIs f\"ur die Benutzerauthentifizierung und IO abh\"angig. Ein von einem Angreifer kontrollierter Host kann die IO-Daten des Benutzers nicht nur beobachten, sondern auch unbemerkt \"andern. Der Verlust der Integrit\"at und Vertraulichkeit von Benutzereingaben kann zu einem katastrophalen Ausfall kritischer Infrastrukturen, zum Verlust von Menschenleben und zum Verlust sensibler Daten f\"uhren. Das Problem der sicheren Kommunikation zwischen einem Benutzer und einem Endsystem wird als \emph{Trusted Path} bezeichnet. Solche Angriffe sind nicht weit hergeholt, da moderne Soft- und Hardwaresysteme unglaublich komplex sind und aus Millionen von Codezeilen bestehen. Daher m\"ussen die Benutzer einer riesigen Trusted Computing Base oder TCB vertrauen. Das Ausnutzen von Software-Schwachstellen in Betriebssystemen, Hypervisoren oder Datenbanksystemen ist weit verbreitet. Neuere Technologien wie Trusted Execution Environments (TEEs) l\"osen dieses Problem durch eine Reduktion der TCB, indem isolierte Umgebungen, die als Enklaven bezeichnet werden, auf CPU-Kernen laufen und vom Betriebssystem oder Hypervisor isoliert sind. TEEs l\"osen jedoch nicht das Problem des Trusted Path, da TEEs f\"ur die Kommunikation mit externen Ger\"aten vom Betriebssystem abh\"angig sind. Dar\"uber hinaus ist der Mechanismus f\"ur Remote Attestation, durch den ein Verifizierer sicherstellen kann, dass er mit der richtigen Enklave kommuniziert, anf\"allig f\"ur Relay-Attacken. Im Kontext einer disaggregierten Rechnerarchitektur in modernen Rechenzentren sind die Sicherheitseigenschaften von traditionellen TEEs unzureichend, da die Trusted-Path-Anwendung sensible Daten nicht nur auf den CPU-Kernen, sondern auch auf spezialisierter externen Hardware, wie z.B. Beschleuniger, beinhaltet.


In dieser Arbeit schlagen wir Mechanismen vor, um Vertrauen in moderne Computerplattformen aufzubauen, indem wir das Trusted-Path-Problem angehen. Dabei leisten wir die folgenden Beitr\"age: Zuerst analysieren wir bestehende Trusted-Path-Systeme und decken mehrere Angriffe auf, die die Integrit\"at und Vertraulichkeit von Benutzer-IO-Daten gef\"ahrden. Wir sind die ersten, die das Trusted-Path-Problem analysieren, um eine Reihe wesentlicher Sicherheitseigenschaften zu finden und diese in einem System namens ProtectIOn unter Verwendung eines vertrauensw\"urdigen eingebetteten Ger\"ats als Intermedi\"ar zu implementieren. Dieses vertrauensw\"urdige Ger\"at f\"angt alle IO-Daten ab und \"uberlagert eine sichere Benutzeroberfl\"ache auf dem Anzeigesignal. Als n\"achstes schauen wir uns Intel SGX an und untersuchen, wie man eine Trusted-Path-L\"osung in TEEs integrieren kann. Wir stellen fest, dass die Relay-Attacke auf die Remote Attestation von SGX die Sicherheitseigenschaften des vertrauensw\"urdigen Pfads beeintr\"achtigen kann. Wir entwickeln ProximiTEE, ein System, das Distance Bounding verwendet, um die physische N\"ahe zu einem SGX-Prozessor zu \"uberpr\"ufen. Wir zeigen auch, wie der Mechanismus des Distance Bounding in hoher Frequenz verwendet werden kann, um Plattformen in Rechenzentren zuzuordnen oder zu widerrufen, ohne auf eine Online-PKI angewiesen zu sein. Schlie{\ss}lich betrachten wir das disaggregierte Rechnermodell moderner Rechenzentren, in denen TEEs nicht ausreichen, da die Berechnung nicht mehr auf CPU-Kerne beschr\"ankt ist, sondern mehrere externe Ger\"ate wie GPUs, Beschleuniger usw. einschliessen. Basierend auf der RISC-V Architektur pr\"asentieren wir unser System PIE, das die Enklaven, die in der CPU und in der externen Firmware-Hardware ausgef\"uhrt werden, zu einer einzigen attestierbaren Dom\"ane kombiniert, die wir plattformweite Enklaven nennen. Innerhalb dieser plattformweiten Enklaven kann f\"ur einzelne Bin\"ardateien (Enklaven und Firmware) Remote-Attestation durchgef\"uhrt werden.


\end{otherlanguage}

\endgroup

\vfill