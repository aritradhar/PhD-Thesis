%*******************************************************
% German Abstract
%*******************************************************

\cleardoublepage%

\begingroup
\let\clearpage\relax
\let\cleardoublepage\relax
\let\cleardoublepage\relax

\begin{otherlanguage}{ngerman}
\pdfbookmark[1]{Zusammenfassung}{Zusammenfassung}
\chapter*{Zusammenfassung}

User interfaces (UI) sind wesentliche Bestandteile moderner komplexer Computerplattformen, da sie vorschreiben, wie Menschen Eingaben in diese Systeme bereitstellen und Ausgaben von ihnen interpretieren. Viele sicherheitskritische und sicherheitskritische Cyber-Physical-Remote-Systeme wie industrielle SPS (in der Fertigung, Kraftwerke usw.) und medizinische Implantate sind �ber umfangreiche Benutzeroberfl�chen �ber Browser oder dedizierte Anwendungen zug�nglich, die auf handels�blichen Systemen oder Hosts ausgef�hrt werden. In �hnlicher Weise sind E-Banking, E-Voting, soziale Netzwerke und viele andere Remote-Anwendungen und -Dienste entscheidend von UIs f�r die Benutzerauthentifizierung und IO abh�ngig. Ein von einem Angreifer kontrollierter Host kann die IO-Daten des Benutzers nicht nur beobachten, sondern auch unbemerkt �ndern. Der Verlust der Integrit�t und Vertraulichkeit von Benutzereingaben kann zu einem katastrophalen Ausfall kritischer Infrastrukturen, zum Verlust von Menschenleben und zum Verlust sensibler Daten f�hren. Das Problem der sicheren Kommunikation zwischen einem Benutzer und einem Endsystem wird als \emph{trusted path} bezeichnet. Solche Angriffe sind nicht weit hergeholt, da moderne Soft- und Hardwaresysteme unglaublich komplex sind und sich �ber Millionen von Codezeilen erstrecken. Daher m�ssen die Benutzer einer massiven Trusted Computing Base oder TCB vertrauen. Das Ausnutzen von Software-Schwachstellen des Betriebssystems, Hypervisoren, Datenbanksystemen ist weit verbreitet. Neuere Technologien wie Trusted Execution Environments (TEEs) l�sen dieses Problem, indem sie den TCB reduzieren, indem isolierte Umgebungen auf den CPU-Kernen laufen, die als Enklaven bezeichnet werden und vom Betriebssystem oder Hypervisor isoliert sind. TEEs l�sen jedoch nicht das Problem des vertrauensw�rdigen Pfads, da TEEs vom Betriebssystem abh�ngig sind, um mit den externen IO-Ger�ten zu kommunizieren. Dar�ber hinaus ist der Fernbeglaubigungsmechanismus, durch den ein Verifizierer sicherstellen kann, dass er mit der richtigen Enklave kommuniziert, anf�llig f�r Weiterleitungsangriffe. Im Kontext einer disaggregierten Rechenarchitektur in modernen Rechenzentren sind die Sicherheitseigenschaften von traditionellem TEE unzureichend, da die Trusted-Path-Anwendung sensible Daten nicht nur auf den CPU-Kernen, sondern auch auf der spezialisierten externen Hardware wie Beschleunigern beinhaltet.


In dieser Arbeit schlagen wir Mechanismen vor, um Vertrauen in moderne Computerplattformen aufzubauen, indem wir das Trusted-Path-Problem angehen, und wir leisten die folgenden Beitr�ge. Zuerst haben wir bestehende Trusted-Path-Systeme analysiert und mehrere Angriffe gefunden, die die Integrit�t und Vertraulichkeit von Benutzer-IO-Daten gef�hrden. Wir sind die ersten, die das Trusted-Path-Problem analysieren, um eine Reihe wesentlicher Sicherheitseigenschaften zu finden und diese in einem System namens ProtectIOn unter Verwendung eines vertrauensw�rdigen eingebetteten Ger�ts als Vermittler zu implementieren. Dieses vertrauensw�rdige Ger�t f�ngt alle IO-Daten ab und �berlagert eine sichere Benutzeroberfl�che auf dem Anzeigesignal. Als n�chstes schauen wir uns Intel SGX an und untersuchen, wie man eine Trusted-Path-L�sung in TEEs integrieren kann. Wir stellen fest, dass der Relay-Angriff auf die SGX-Remote-Best�tigung die Sicherheitseigenschaften des vertrauensw�rdigen Pfads beeintr�chtigen kann. Wir entwickeln ProximiTEE, ein System, das Distanzbegrenzung verwendet, um die physische N�he zu einem SGX-Prozessor zu �berpr�fen. Wir zeigen auch, wie der Distance Bounding-Mechanismus in hoher Frequenz verwendet werden kann, um Plattformen in Rechenzentren zuzuweisen oder zu widerrufen, ohne auf eine Online-PKI angewiesen zu sein. Schlie�lich betrachten wir das disaggregierte Rechenmodell moderner Rechenzentren, in denen die TEEs nicht ausreichen, da die Berechnung nicht mehr auf die CPU-Kerne beschr�nkt ist, sondern mehrere externe Ger�te wie GPUs, Beschleuniger usw. Wir schlagen unser System PIE basierend auf RISC-V-Architektur, die die Enklaven, die auf der CPU und der externen Firmware-Hardware ausgef�hrt werden, zu einer einzigen attestierbaren Dom�ne kombiniert, die wir plattformweite Enklaven nennen. Innerhalb dieser plattformweiten Enklaven k�nnen einzelne Bin�rdateien (Enklaven und Firmware) aus der Ferne best�tigt werden.

\end{otherlanguage}

\endgroup

\vfill