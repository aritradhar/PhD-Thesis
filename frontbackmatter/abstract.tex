%*******************************************************
% Abstract
%*******************************************************
%\renewcommand{\abstractname}{Abstract}
\pdfbookmark[1]{Abstract}{Abstract}
\begingroup
\let\clearpage\relax
\let\cleardoublepage\relax
\let\cleardoublepage\relax

\chapter*{Abstract}

User interaction is an essential part of modern complex computing platforms as it dictates how humans provide inputs to these systems and interpret output from them. In the last several decades, there has been massive progress in how the user communicates with software systems starting from a punch card to today's state-of-the-art AR/VR. Many remote safety-critical systems such as industrial PLCs (in manufacturing, power plants, etc.), medical implants can be accessed through the browser or dedicated application that provides rich user interfaces. Apart from these cyber-physical systems, e-banking, e-voting, social networks, and many more remote applications and services are dependent on UI for both authentication and user IO. An attacker-controlled platform can not only observe user's IO data but also can modify them undetected. Such attacks could be catastrophic as the loss of integrity and confidentiality of user inputs may lead to loss of critical infrastructure, human lives, leakage of sensitive data -- known as the problem of the \emph{trusted path}. Such attacks are not far-fetched as modern software and hardware systems are incredibly complex and span over millions of lines of code; hence one needs to trust the massive software trusted computing base or TCB. Exploiting software vulnerabilities of OS, hypervisors, database systems are very prevalent. Recent technologies such as Trusted execution environments (TEEs) address this problem by reducing the TCB by running enclaves isolated from the OS. However, TEEs do not solve the trusted path problem as TEEs depend on the OS to communicate to the external IO devices. Moreover, in the context of disaggregated computing architecture in modern data centers, the traditional TEE is insufficient as the trusted path application involves sensitive data not only on the CPU cores but also on the specialized external hardware like accelerators.


In this thesis, we address the trusted pat problem in modern platforms and make the following contributions. First, we analyze existing trusted path systems and found several attacks that compromise user IO data integrity and confidentiality. We are the first to analyze the trusted path problem to find a set of essential security properties and implement them in a system named ProtectIOn using a trusted embedded device as an intermediary. This trusted device intercepts all IO data and overlays secure UI on the display signal. Next, we look into Intel SGX and investigate how one can integrate a trusted path solution to TEEs. We notice that the relay attack on the SGX remote attestation can be detrimental to the trusted path security properties. We design ProximiTEE, a system that uses distance bounding to verify physical proximity to an SGX processor. We also show how the distance bounding mechanism can be used in a high frequency to allocate or revoke platforms in data centers without relying on an online PKI. Finally, we look into the disaggregated computing model of the modern data centers where the TEEs are insufficient as the computation is no longer limited to the CPU cores but several external devices such as GPUs, accelerators, etc. We propose our system PIE based on RISC-V architecture that combines the enclaves running on the CPU and firmware external hardware to a single attestable domain that we call platform-wide enclaves. Inside these platform-wide enclaves, individual binaries (enclaves and firmware) can be remotely attested.
 

  


\endgroup

\cleardoublepage%

\begingroup
\let\clearpage\relax
\let\cleardoublepage\relax
\let\cleardoublepage\relax

\begin{otherlanguage}{ngerman}
\pdfbookmark[1]{Zusammenfassung}{Zusammenfassung}
\chapter*{Zusammenfassung}

Deutsche Zusammenfassung hier (German abstract here).

\end{otherlanguage}

\endgroup

\vfill