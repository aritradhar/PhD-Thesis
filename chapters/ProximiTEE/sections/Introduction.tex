\section{Introduction}
\label{sec:intro}


In the previous chapter, we discussed \protection, where we investigate the fundamental security properties required by a trusted path. This section shows that porting a trusted path application into a trusted execution environment such as Intel SGX is non-trivial. Trusted execution environments (TEEs) like Intel's SGX enable securely execute applications on untrusted computing platforms. SGX enables protected applications that are called enclaves that are isolated from the operating system. The CPU microcode enforces the isolation mechanism. Remote attestation is a key feature of SGX and other similar TEE architectures. It allows a remote verifier to check that the attested enclave was correctly constructed before provisioning secrets to it. More details about SGX attestation is provide in Section~\ref{ch:background:SGX:attestation}.


\subsection{Relay attacks}  While SGX's remote attestation guarantees that the attested enclave runs the expected code, it \emph{does not}, however,  guarantee that the enclave runs on the expected computing platform. An adversary that controls the OS (or other software) on the target platform can relay incoming attestation requests to another platform. Such relay attacks are a long-standing open problem in trusted computing, as already a decade ago, Parno identified such attacks in the context of TPM attestation~\cite{parno2008bootstrapping}.


Upon a first look, it might seem that relay attacks do not pose a problem for TEEs. If the attacker relays the attestation to another machine, the same security guarantees should hold since the data will only be available within the remote TEE, and the enclave code that can access the provisioned secrets is verified. However, such simple reasoning is incorrect. 

In this chapter, we provide the first careful analysis of the \emph{implications} of relay attacks on SGX and show that by relaying, the adversary increases his capabilities to attack the attested enclave significantly. One example of increased adversarial capabilities is physical side-channel attacks. If the adversary redirects the attestation to a platform that he physically controls, he can mount various physical side-channel attacks, like ~\cite{genkin2016physical,wang2006covert,gandolfi2001electromagnetic, shamir2004acoustic}, that would not have been possible without the relay. Another example is enhanced side-channel attacks. While controlling the OS is in the SGX attacker model, it is not unrealistic that an attacker might be in the situation of controlling only user-privileged code on the target platforms. This degree of control, however, allows the attacker to redirect attestation to another platform where he controls the OS, which allows him to launch software-based side-channel attacks, such as~\cite{moghimi2017cachezoom, sgxcache, gotzfried2017cache}, that leverage system privileges to attack enclaves. In Section~\ref{sec:problemStatement}, we explain further examples of attacks that are enabled by attestation relay.



A typical ``solution'' to relay attacks is to assume \emph{trust on first use} (TOFU). However, in many application scenarios, TOFU is neither secure nor practical. For example, solutions where attestation is performed immediately after a fresh OS installation cannot be applied to settings where OS reinstallation is simply not possible. Besides, all TOFU variants assume that the target platform OS is trusted, even if momentarily, which violates SGX's trust model. %Trust on first use also has other, more subtle, problems that we explain in Section~\ref{sec:problemStatement}.



The SGX attestation protocol is designed to be anonymous. The protocol is based on EPID group signatures~\cite{epid_attestation}, and thus the remote verifier cannot distinguish whether the correct enclave on the target platform was attested or if the attestation was redirected to another platform. Upon first inspection, it may seem like relay attacks are only possible because of such anonymity features and that relaying could be easily prevented if attestation protocols were designed to be non-anonymous. However, such simple reasoning is incorrect as well. 
%
We show that all SGX attestation variants, including the ``linkable'' attestation mode and the recently introduced Data Center Attestation Primitives (DCAP)~\cite{DCAP} are vulnerable to relay attacks. We also explain why relay attacks would remain possible, even if all anonymity features would be removed from the attestation.


\subsection{VM pinning and Revocation} 

Modern data centers provides flexible resource management for optimal allocation of resources such as storage, memory, network interface, etc. Clients are provided with virtual machine (VM) instances and more than one client VMs can run on a same physical platform. Due to the flexible nature of resource allocation, the VMs are often not physically bound to a specific platform. I.e., a data center operator can dynamically allocate a piece of its physical infrastructure and migrate the VM. This ``elastic'' nature of VM enable the data center operator to claim more physical infrastructure (CPU cores, DRM, NIC, disk) as the client demand increases. However, in many applications where the sensitive nature of the data is critical, it's necessary to ensure that the user's VM stays bound to a physical platform. Such property is highly desirable in many application scenario. E.g., a client would like to keep all her banking data inside her country of residence, or regulations such as GDPR may mandate the operators to store the user data in a specific country. This is particularly challenging when considering TEEs like Intel SGX due to its lack of platform identity. Aside from the relay attack, revocation of a physical platform is a challenging aspect of data centers. Revocation enables the data center operators to revoke a certain set of physical platforms either for maintenance or discovery of a vulnerability.

\subsection{Our solution} 

We propose a new solution, called \name, that prevents relay attacks by leveraging a simple embedded device that is attached to the attested target platform. Our solution is best suited to scenarios where i) the deployment cost of such an embedded device is minor compared to the benefit of more secure attestation, and ii) TOFU solutions are not acceptable. Attestation of servers at cloud computing platforms and setup of SGX-based permissioned blockchains are two such examples. 

In \name, the remote verifier establishes a secure connection to the embedded device whose public key it knows through standard device certification. The device performs normal SGX attestation and additionally \emph{verifies the proximity} of the attested enclave using a simple distance-bounding protocol~\cite{distanceBounding}. After the initial attestation, the device performs \emph{periodic} distance-bounding measurements, and the communication channel created during the attestation stays active only as long as the device is connected to the same platform. Thus, the physical act of attaching the device to an SGX platform enables secure attestation (enrollment), while detaching the device will prevent further communication with the attested enclave (revocation). Neither enrollment nor revocation requires interaction with a trusted authority. This property is useful in applications like permissioned blockchains, where validator nodes are separate organizations assigned by a trusted authority. The authority can issue one device per organization, and each organization is free to manage their computing resources (e.g., detach the device from one platform and attach it to another) without interaction with the authority. 


\subsection{Main results.} Parno~\cite{parno2008bootstrapping} identified distance bounding as a candidate solution to TPM relay attacks already ten years ago, but concluded that it could not be realized securely as the slow TPM identification operations (signatures) make a local and relayed attestation indistinguishable. Our evaluation shows that proximity verification \emph{is} possible for SGX, assuming very fast adversaries. The main reason why distance bounding protocols work for SGX, but not with TPMs, is that SGX is a programmable TEE where it is possible to use pre-established security associations and efficient challenge-response protocols based on simple operations such as \texttt{XOR}.

To evaluate \name, we implemented it using a USB 3.0 prototyping board. The main purpose of our evaluation is to demonstrate that the adversary cannot redirect the attestation \emph{over the Internet} to an adversary-controlled platform without being detected. We focus on such relay, as it offers the most increased capabilities to the adversary (e.g., physical attacks). The secondary purpose of our evaluation is to determine whether proximity verification can prevent relay to a co-located platform, like another server on the same server rack. Such relays are typically less harmful, but ideally, they should be prevented as well.



In our evaluation, we \emph{simulate} a strong adversary that i) is only a single network hop away from the target, ii) performs the required protocol computations instantaneously, iii) has infinitely fast hardware interface, and iv) has enabled software-based packet forwarding optimizations on the target platform. We measure the legitimate challenge-response latency on our prototype to be $185 \mu s$ on average. In the case of the simulated relay attack, the average latency is about $264 \mu s$. These two latency distributions are distinguishable and allow us to set our proximity verification protocol parameters such that the adversary's probability of performing a successful relay attack is negligible ($3.55\times 10^{-34}$), while legitimate verification succeeds with a very high probability ($0.999999977$). Importantly, the adversary cannot increase his success probability with repeated attempts, as attestation is triggered by the trusted remote verifier. Our experiments also show that enclave revocation using periodic proximity verification is both secure and practical.


The performance overhead of proximity verification is small: the initial
proximity verification adds only a small delay to the attestation protocol, and
the periodic proximity verification consumes only a very minor fraction of
the available \usb 3.0 channel capacity. Our implementation shows that the complexity of such a device can be small: the software TCB of our prototype is 3.8 KLoC.


\myparagraph{Emulation attacks.} Additionally, we consider a stronger adversary that has obtained leaked, but not yet revoked, attestation keys and can \emph{emulate} an SGX-enabled processor.
%
Proximity verification alone cannot prevent emulation attacks, as a perfectly emulated enclave would pass any proximity test. Therefore, we propose a second attestation mechanism based on \emph{boot-time initialization}. 

In this solution, the target platform loads a small, single-purpose kernel from the attached device and launches an enclave that seals a secret key known by the device. 
%After this initialization, the target platform can reboot onto the regular OS. 
Subsequently, when attestation is needed, the enclave can verify the proximity of other enclaves on the same platform using SGX's local attestation. This enables secure attestation regardless of potentially leaked attestation keys. Our second solution can be seen as a novel variant of the well-known TOFU principle. The main benefits over previous variants are easier adoption (e.g., no OS reinstallation) and increased security (e.g., OS not trusted even temporarily).


\subsection{Contributions} This chapter contributions are organized as follows:

\begin{enumerate}
    \item \emph{Analysis of relay attacks.} While relay attacks have been known for more than a decade; their implications have not been fully analyzed. In Section~\ref{sec:problemStatement}, we provide the first such analysis and show how relaying amplifies the adversary's capabilities for attacking SGX enclaves.   

    \item \emph{\name: Addressing relay attacks.} In Section~\ref{sec:systemDesignMain}, we propose a hardened SGX attestation mechanism based on an embedded device and proximity verification to prevent relay attacks. \name does not rely on the common TOFU assumption, and hence, our solution improves the security of previous attestation approaches. Note that the distance bounding approaches are well-known in the literature, but using such a method in the context of SGX is non-trivial.
    
    \item \emph{Experimental evaluation.} We implement a complete prototype of \name and evaluate it against a very strong and fast adversary. Our evaluation in Section~\ref{sec:evaluation} is the first to show that proximity verification can be both secure and reliable for TEEs like SGX.
    
    \item \emph{Addressing emulation attacks.} We also propose another attestation mechanism based on boot-time initialization to prevent emulation attacks. This mechanism, described in Section~\ref{sec:variantII}, is a novel variant of TOFU with deployment, security, and revocation benefits.
\end{enumerate}

\subsection{Organization of this Chapter}

The rest of this chapter is organized as the following. Chapter~\ref{sec:problemStatement} describes the problem of relay attack and the trust on first use (ToFU) properties of all the variants of Intel SGX remote attestation. In Section~\ref{sec:systemDesignMain}, we address the relay attack by using the distance bounding protocol using a trusted embedded device. Section~\ref{sec:security_proximitee},~\ref{sec:implementation} and~\ref{sec:evaluation} describes the security analysis, implementation and evaluation of the distance bounding mechanism. In Section~\ref{sec:variantII} we discuss the emulation attacker, which is stronger than the relay attacker as the attacker has access to at least one leaked attestation key. In this section, we describe the bootstrap mechanism to address the emulation attacker along with the security analysis and implementation of the mechanism. In Section~\ref{sec:secureInput}, we provide a system design that enable trusted path using \proximitee. Section~\ref{sec:discussion_proximitee} provides additional discussion, and Section~\ref{sec:conclusion_promimitee} concludes this chapter.



