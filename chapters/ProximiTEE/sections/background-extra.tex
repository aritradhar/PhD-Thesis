%!TEX root =  ../paper.tex
\section{SGX Background}
\label{sec:background-extra}

In this appendix we provide additional SGX background.


\myparagraph{Limitations and vulnerabilities.} Recent research has demonstrated that the SGX architecture can be susceptible to side-channel leakage. Secret-dependent data and code access patterns can be observed by monitoring shared physical resources such as CPU caches~\cite{sgxcache,gotzfried2017cache,moghimi2017cachezoom} or the branch prediction unit~\cite{lee2017inferring}. The OS can also infer enclave's execution control flow or data accesses by monitoring page fault events~\cite{xu2015controlled}. Many such attacks can be addressed by hardening the enclave's code, e.g., using cryptographic implementations where the data or code access patterns are independent of the key.

The recently discovered vulnerabilities Spectre~\cite{Kocher2018spectre} and Meltdown~\cite{Lipp2018meltdown} allow application-level code to read memory content of privileged processes across separation boundaries by exploiting subtle side-effects of speculative execution. The Foreshadow attack~\cite{foreshadow-usenix18} demonstrates how to extract SGX attestation keys from processors by leveraging the Meltdown vulnerability. 


\myparagraph{Microcode updates.}
During manufacturing each SGX processor is equipped with hardware keys. When SGX software is installed on the CPU for the first time, the platform runs a provisioning protocol with Intel. In this protocol, the platform uses one of the hardware keys to demonstrates that it is a genuine Intel CPU running a specific microcode version and it then then joins a matching EPID group and obtains an attestation key~\cite{epid_attestation}. 

Microcode patches issued by Intel can be installed to processors that are affected by known vulnerabilities such as the above mentioned Foreshadow attack. When a new microcode version is installed, the processor repeats the provisioning procedure and joins a new EPID group that corresponds to the updated microcode version and obtains a new attestation key which allows IAS to distinguish attestation signatures that originate from patched processors from attestation signatures made by unpatched processors~\cite{epid_attestation}.
