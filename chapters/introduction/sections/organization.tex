\section{Thesis Organization}

The thesis is organized in five parts and six chapters (including this chapter) as the following: 

Part~\ref{part:intro} contains Introduction (Chapter~\ref{ch:introduction}), and Background (Chapter~\ref{ch:background}). Chapter~\ref{ch:introduction} introduces the problem statement, research questions that this thesis addresses and a brief overview of the contribution of this thesis. Chapter~\ref{ch:background} provides a brief overview of trusted path, Intel SGXand RISC-V keystone that serves as the background of this thesis.

Part~\ref{part:howNot} consists of Chapter~\ref{ch:integrikey} and~\ref{ch:integriscreen} that provide details of the two trusted path systems based on two different second-factor methods:\integrikey and \integriscreen. \integrikey is based on input signing, and \integriscreen is based on the visual verification of user intent. In Part~\ref{part:fundamental}, Chapter~\ref{ch:protectIOn}, we discussed \protection that explores fundamental security properties of the remote trusted path and shows attacks on existing trusted path solutions. 

Next, Part~\ref{part:relay} contains Chapter~\ref{ch:proximitee}, where we discuss the challenges with relay attacks when we try to port an existing trusted path application to a TEE like Intel SGX, and we address this by introducing a new system based on distance-bounding mechanism: \proximitee. \proximitee shows how one can compromise a trusted path application without even compromising any security properties of the trusted path itself, but only by compromising underlying TEE properties - remote attestation. 

Chapter~\ref{ch:pie} in Part~\ref{part:platform} describes \pie, where we address the new challenges that arise from the disaggregated computing model where the trusted path application is not only bounded to the CPU cores, rather distributed among different hardware components of the platform such as a GPU, TPU, NIC, etc. 

Finally, Chapter~\ref{ch:summary} in Part~\ref{part:conclusion} summarizes this thesis and provides concluding remarks.
