\section{Security Analysis}
\label{sec:securityAnalysis}

\subsection{Trust assumption}
\label{sec:securityAnalysis:trustAssumption}

The trust assumption can be different based on different scenario.
Table~\ref{table:trustAssumption} describes different trust assumption when
different privacy properties are required. For example remote authentication
and confidentiality is required for user credential for critical applications
such as online banking. There also exists numerous scenarios where input
integrity is critical such as the amount of money to be sent by internet banking
application. Here one thing is to be noted that the combination of the
properties: remote authentication and input confidentiality provides security
against any type of phishing attack. These phishing attacks include both remote
and local attack where the later assumes that the host system is malicious.


\subsection{Detailed attacker model}
\label{sec:securityAnalysis:attackerModel}

We consider a global, adaptive and active attacker. The global aspect implies
that the attacker has a full view of the system, i.e., he can compromise the
user-side client (the host system) and the network (ISP level attacker). He is
also adaptive in the sense that based on the user interaction he can modify his
strategy and active means the attacker can listen, delay, modify and drop
messages. In summary we consider a Dolev-Yao attacker model. The attacker's goal
here is three-fold.

\begin{enumerate}
\item The attacker impersonates a legitimate website to make the user believe
that the user is accessing the legitimate website. This allows the attacker to
steal sensitive information (such as the voting preferences and other personal
sensitive data). We consider this attack to be orchestrated in two ways:
remotely and locally. The remote attack is the classical phishing attack where
the attacker sits outside the host system and impersonates a legitimate website
by either making a very small change in the website name or having compromised
the browser (\ad{cite the attack here}). The local attack however is more
sophisticated as the attacker compromises the host system which spoofs the
security indicator, rendering a malicious website as a legitimate one.

\item The attacker goal here is to extract as much sensitive information as
possible from the user. This includes login credentials, second factor codes,
other sensitive personal data etc. One can regard this attacker as the software
or hardware key-logger that records all keyboard strokes and mouse clicks.

\item We also assume that the attacker is capable of executing side
channel attacks on the secure hub such as timing leakage and power analysis to
extract additional information about the sensitive information the user
provides.
\end{enumerate}

\ad{What the attacker is capable of doing?}

We consider different levels of system compromise and list three active
attackers with varying abilities to reflect different trust assumptions. We
assume that the peripheral operated by the user is fully trusted. We classify
the attacker in two broad classes:

\subsubsection{Remote attacker}
%\subsubsection{Compromised Network}
Here we fully trust the host system. The network is fully compromised and the
attacker here is remote. For example, it can mount a phishing attack to
trick the user into entering login credentials. The attacker can also eavesdrop,
insert, and delete messages in the channel between the website and browser.


\subsubsection{Local attacker}

The local attacker is more powerful compared to the remote attacker. Apart from
possessing all the capabilities of the remote attacker the local attacker
additionally can compromise part of the host system in the following ways:

\begin{itemize}
  \item \textbf{Compromised Browser.}
Here we assume that the browser is compromised, while the underlying operating
system is still trusted. As a result, the malicious browser observes all
sensitive user input and can display arbitrary content to the user. For example,
the browser may fake the security indicators to show that a phishing website is
legitimate to the user.

\item \textbf{Fully compromised host.}
This is the strongest attacker model where the adversary fully compromises the
host system (operating system and hardware). Any display rendered by the
operating system can not be trusted.
\end{itemize}

