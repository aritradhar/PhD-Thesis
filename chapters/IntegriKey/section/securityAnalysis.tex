\section{Security Analysis}
\label{sec:securityAnalysis_IK}

In this section, we provide an informal security analysis of our system. The goal of the attacker is to violate the \emph{integrity} of web input. Examples include a misconfiguration of a safety-critical device or a false payment against the intention of the user. 

\subsection{Arbitrary Modifications} 

The simplest adversarial strategy is to manipulate only the application payload that is sent to the server. The attacker can, e.g., change one input value provided by the user to another arbitrary value in the HTTP response. Such attacks are detected by the server because the configuration data received over $TLS_1$ does not match the traces received over $TLS_2$.


\subsection{Swapping Attacks}
%\label{sec:analysis:swapping}

A more sophisticated adversarial strategy is to manipulate both the application payload and the user interface at the same time. More specifically, the attacker can change the names and the order of the user input fields and modify any instructions that are part of the user interface, such as the labeling instructions. Figure~\ref{fig:swapExample} shows one such example of the attack where the field names `\texttt{Relay temp 1}' and `\texttt{Relay temp 2}' are swapped.


The goal of a swapping attack is that the server interprets the received input values with different semantics than the user intended. Assuming that the user interface contains interchangeable fields, the attacker can construct an HTTP response where all input values are listed in the correct order, and their values match the input events. Two variants of such attacks are possible. 

\myparagraph{Manipulated field names or instructions} 
\label{sec:securityAnalysis_IK:cases}
In the first variant, the attacker manipulates \emph{either} the field names \emph{or} the instructions. In such a case, the user interface that is shown to the user has an \emph{inconsistency} because the input field names and the labeling instructions do not correspond to each other. The user may react in different ways that we enumerate below:
%\vspace{-5pt}
\begin{itemize}
    \item \emph{Case 1: Abort.} The user may notice the inconsistency in the UI and abort the process. \label{detect}
    
    \item \emph{Case 2: Correct labeling.} The user may perform the labeling correctly. That is, he may prefix each entered input value with the matching label. The target device is configured correctly, despite the user interface manipulation.\label{notdetect}
    
%    \item \textbf{Case 3: Correct labeling (UI).} The user may label each input based on the manipulated UI and against the labeling instructions. The server will detect the user interface manipulation because the user inputs are received in the wrong order and abort the configuration process. \todo{Hmmm... confusing... merge 2 and 3?}\label{corrLable}
    
    \item \emph{Case 3: Incomplete labeling.} The user may fail to complete the required labels. The server will abort the process. 
    
    \item \emph{Case 4: Incorrect labeling.} Finally, the user may perform the labeling incorrectly. That is, he may associate one of the asked input values with an incorrect (and swappable) label prefix. The server cannot detect this case, and the target device will be misconfigured. \label{kaboom}
\end{itemize}
%\vspace{-8pt}

In Section~\ref{sec:results:userStudy} we report results of a small-scale user study that provides preliminary evidence on how common these cases are, and especially how many people would fall for the attack (\emph{Case 4}).


\myparagraph{Manipulated field names and instructions} 
In the second variant, the attacker manipulates \emph{both} the field names and the labeling instructions. In this variant, there are two possible cases. First, the labeling instructions do not correspond to the UI field order, in which case the effect is the same as above. Second, the modified labeling instructions correspond to the modified UI, i.e., the matching field names and labeling instructions are both modified the same way. In this case, the labeling instruction reordering essentially nullifies the effect of UI field name reordering, and the UI is consistent again (no risk of misconfiguration).


We conclude that any manipulation of (\emph{i}) UI field names, (\emph{ii}) values in the application payload, (\emph{iii}) labeling instructions in the UI, or (\emph{iv}) the combination of thereof cannot violate input integrity unless there is a visible indication of it (i.e., inconsistency) in the user interface. This is the \emph{Case 4} above, which we evaluate with a small user study.



\subsection{Privacy Considerations} 
\label{sec:privacy}

In our solution, the trusted \device device intercepts the user's keyboard input events and sends a trace of them to the server for matching. As \emph{any} interception-based solution has obvious privacy concerns, in this section, we explain why the typical and recommended usage of our solution does not violate user's privacy.


\begin{enumerate}
	\item \emph{Device removal.} The primary usage model for our solution is one where the user attaches the \device device before a security-critical web operation and removes it after it. Thus, in such typical use, user input events outside the security-critical operation are not shared with the server. We do not recommend using our solution as a generic input protection mechanism for all web browsing but rather as a hardening mechanism for only specific security-critical operations.

	\item \emph{Server white-listing.} The \device sends the user input only to one (or more) pre-authorized (white-listed) servers. Therefore, even if the user would forget to remove the \device device from the host after the security-critical operation, the user input would be shared only with known and trusted servers. Such servers can implement additional privacy-preserving mechanisms like send a signal to the device to stop input sharing once the operation is completed.

	\item \emph{Safe handling of tabs.} As the \device uses \webusb as the communication channel through the host's browser, the implementation of the \webusb restricts only one website (known as the \emph{landing page}) to bind with the \usb device~\cite{webuseb_google}. This ensures that if the user switches the browser tab \emph{during the security-critical operation}, the keystrokes from that application will not be forwarded to the \device unless the user reinitializes the device manually.
\end{enumerate}

Finally, we emphasize that even in the worst case where the user forgets to remove the device and one of the white-listed servers turns out to be malicious. Our solution does not \emph{reduce} the user's privacy when compared to use without our solution. Since we assume a compromised host, the OS can trivially share all user input with any server regardless of whether our solution is used or not.



\subsection{Other Security Considerations}

\myparagraph{Default values} \name eliminates any default values on the webpage. However, the host can always show default values to the user. If the user does not type the value by herself, the server rejects the input as the data from the browser, and the trace does not match.

\myparagraph{Trace dropping} Since all communication from \device to the server is mediated by the untrusted host; the attacker may also attempt to manipulate the traces by selectively dropping packets (e.g., remove certain user input). However, such attacks are prevented by the use of a standard TLS connection.

\myparagraph{Cross-device attacks} An additional attack strategy is to trick the user into providing input for the configuration of one safety-critical device but using this user input for the configuration of another device. In such cross-device attacks, the host presents to the user the configuration user interface from server A but tricks \device to establish a connection with server B.

Cross-device attacks are only possible if (\emph{i}) the same \device is pre-configured for both servers A and B, (\emph{ii}) every user input field in the configuration web pages of servers A and B is interchangeable, and (\emph{iii}) both configuration pages have exactly the same labels. We consider such cases rare. 
To protect against cross-device attacks, both configuration user interfaces A and B can be processed with the same instance of \tool, which can annotate the pages with unique labels.




