\section{Background}

\subsection{USB}
\ad{talk about USB here and that all the peripherals are driven by USB. USB provide both data and power and make it ubiquitous}

\subsection{Browser Security}

\dy{We need a background to motivate the work and discuss the security model of browsers (sandboxing, permission delegation...) and mention the attack vectors (zero-days, etc)}

\subsection{WebUSB}

\dy{Please describe the WebUSB browser/OS mechanisms we rely on in this work.}
The Universal Serial Bus (USB) is the de-facto standard for wired (even some
wireless) peripherals.
\webusb\ is an emerging standard developed primarily bu Google which enables
browsers to directly communicate with USB devices. \webusb API safely exposes
USB functionalities where a Javascript code served by a HTTPS page can query and
communicate with USB devices and acts as an USB endpoint.
\ad{I will provide more details later on}

\subsection{Web Bluetooth}

\dy{I will fill in this part}

\subsection{Sensitive Use Cases}

We now motivate this work by addressing some important applications in which
sensitive user data are sent through modern browsers.

\textbf{Secure Communication.}
Users would like to communicate online using web applications in the browser
(e.g., using WebRTC) and discuss sensitive topics with minimal disclosure to any
other party. In such a context, the browser or the hosting website should not be
able to eavesdrop the conversation.

\textbf{Online credentials.}
This is applied to almost all websites which require users to log-in using
username and password. These credentials are extremely sensitive and subject to
phishing attack or extraction via password database leakages.

\textbf{Online Payment.} 
The user makes an online payment by entering credit card details and would like
to be assured that such sensitive information is sent only to the servers of the
respective credit card firms. Otherwise, an online vendor may store the user's
credit card information and abuse it elsewhere.

\textbf{Online Posting.}
The user makes an online posting (e.g., comments to a news article) to express a
personal opinion on sensitive matters, while wishing to remain anonymous from
other parties. If the user's identity is revealed, there may be threats to
personal safety or profile security.

\textbf{Online Voting.}
Electronic voting requires some strong security properties such as: 1. nobody
except the voter knows her vote i.e., the voting decision of the voter remains
anonymous and 2. The counting of the voting must be legitimate. Out of these two
properties the first i.e., the vote casting is done on the browser and the
sensitive data such as voter credential and the vote itself are confidential and
handled by the browser.
