
\section{Conclusion}
\label{sec:conclusion_IK}

Remote configuration of safety-critical systems is prone to attacks where a compromised host modifies user input. Such attacks can have severe consequences that can put human lives in danger. In this paper we have proposed a new solution, called \name, to prevent user input manipulation by the untrusted host. In our scheme, the user installs a simple embedded device between the user input peripheral and the host. This device sends a trace of user input events to the server that can detect input integrity violations by comparing it to the received application payload. Our evaluation shows that \name is cheap to build, easy to deploy, and it works in practice.
%Therefore we consider it an excellent approach to protect systems where remote user input integrity is critical.

%and examined a range of commercially available remote configuration UIs with it. 
%Our input integrity protection adds only a negligible delay to typical remote configuration process. If the target user interface includes swappable elements, our labeling scheme increases user effort slightly. 

%We designed, implemented, and evaluated \name, a system that provides end-to-end integrity for peripheral devices. We identified and analyzed a new form of attack---the UI integrity manipulation attack---that can be orchestrated by a malicious host by manipulating the web page elements shown in the browser. \name preserves the integrity of user input in the presence of such attacks. \tool is very easy to deploy, making it very practical to use in legacy and modern systems alike, such as industrial PLCs, medical programmers, home automation systems, etc. The \device can be used in many client systems and be paired with any \usb based peripheral devices. Our evaluation showed that the \tool server-side component and the \device on the user's side introduce only minimal overhead. Currently, \tool is limited to character-based input devices such as the keyboard. As future work, we aim to explore the integration of other input devices, such as mice, touchscreen and more complex UI elements.