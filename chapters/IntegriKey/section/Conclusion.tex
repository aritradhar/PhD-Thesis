
\section{Conclusion}
\label{sec:conclusion_IK}

The remote configuration of safety-critical systems is prone to attacks where a compromised host modifies user input. Such attacks can have severe consequences that can put human lives in danger. In this chapter, we have proposed a new solution, called \name, to prevent user input manipulation by the untrusted host. In our scheme, the user installs a simple embedded device between the user input peripheral and the host. This device sends a trace of user input events to the server that can detect input integrity violations by comparing it to the received application payload. Our evaluation shows that \name is cheap to build, easy to deploy.

However, despite identifying a new form of UI manipulation attack, and addressing it, \integrikey is still not a fully secure system. As the \device is oblivious to what is shown to the user, it can always be tricked by the attacker by manipulating some instruction or the input shown on the screen. We describe these attacks in \protection described in Chapter~\ref{ch:protectIOn}.
