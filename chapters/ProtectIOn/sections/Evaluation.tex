

\section{Prototype Evaluation}
\label{sec:eval}


%\subsection{Performance}
%\label{sec:eval:performance}

We evaluate the performance of our prototype by measuring the overheads introduced by \name to the system and whether they influence the user's interaction. Initially, we measure the default latency introduced by \device when the user interacts with applications that do not require protection. Table~\ref{tab:performance} provides the relevant latencies \red{and the accuracy of the pointer detection}.
The delay in forwarding keystrokes is $170\ \mu s$, and for frames is $21.76\ ms$. This allows the \device to achieve the maximum display frame rate of $47.69$ per second (e.g., most of the movies are shot and shown in  ~24-30 fps). However, an optimized implementation of the technique to encode information in the HDMI frame would reduce the processing time of a frame significantly and increase further the frame rate as a result. The B101 HDMI to CSI HDMI interceptor has a hardware limit of 25 frames at 1080p resolution. \red{We report 0.997 accuracy of the pointer detection mechanism that involves image analysis and pointer motion tracking. The accuracy is evaluated from 4196 captured frames.}
%\blue{In an hour of usage on our test UIs (simple web forms), we observe that \device re-calibrates the mouse pointer only once.} Note that the accuracy may vary depending on the screen composition.} 
\blue{We observe that the misdetection happens only when the pointer is not completely visible, i.e., the pointer is on the border of the screen and the OS displays it partially. Note that one could improve the logic of \device to run the adjustment phase (see Section~\ref{sec:systemDesign:analysis}) only when the pointer is within the screen completely.}

Our prototype of \name does not require the user to install any additional software in her machine to facilitate the communication between the remote server and the \device. Instead, the \device communicates with the remote server by using the host as an untrusted transporter. Therefore, we start by measuring the delay of sending data from the device to the host and vice versa:


\subsection{\device $\rightarrow$ host} The \device transmits data (encrypted) to the host by simulating keystrokes. In our system, \device sends the keystrokes in a chunk of $256$ bytes of data to the host. The keystroke has an average latency of $5\ ms$, which is undetectable by humans.  


\subsection{Host $\rightarrow$ \device} The host sends data to the device by encoding them into the HDMI frame. The QR-code is generated locally in the browser and displayed on the screen. For a specification of a form with two/four elements, QR-code generation takes $14\ ms$. The \device detects the QR-code, decodes it, and creates the overlay. This process takes $6\ ms$ for the same form considered previously.
 
%\name introduces the following delays in web applications:

\subsection{Initial Page Load} The first time the user visits a web page that employs \name, the remote server, and the \device should exchange a cryptographic key to protect the communication. This step requires only one additional \texttt{xmlHttpRequest} to the server; therefore the delay is relatively low. Initially, the browser encodes the server's public key into a QR-code that is decoded by the \device, which sends the response to the server by simulating the keystrokes.


\subsection{Frame processing for mouse} \device processes every frame of the host for pointer detection. This takes $1.76 ms$, which does not impact the frame rate. The image analysis routine achieves an accuracy of $0.997$. 


\subsection{Keystroke latency} The \device intercepts all user's keystrokes and forwards them to the host or renders on the screen. When rendering on the screen, the latency is $170\ \mu s$.


\subsection{Cursor latency} Similarly to keystrokes, the \device intercepts mouse events also. However, the latency of event forwarding is $250\ \mu s$.


%Key exchange takes around $200$ ms. Frame rate $20-24$ fps. Mouse/keyboard latency \textless$10\ ms$.

%\subsection{\red{Security Evaluation}}
%\label{sec:eval:sec}

\begin{table}[t]
\small
\centering
\begin{tabular}{c |  l | c}
\multicolumn{2}{c|}{\textbf{Projects}} & \textbf{LOC} \\\hline
%&\multicolumn{2}{c}{\textbf{Browsers}}\\\hline
\rowcolor{Gray}&Chromium (Google Chrome)~\cite{chromium_2019} &  $25,163,547$\\
\rowcolor{Gray}\multirow{-2}{*}{Browser} &Mozilla Firefox~\cite{mozilla_2019} & $20,928,358$\\
\multirow{2}{*}{JS Engine}&Chrome V8~\cite{V8} & $2,009,183$\\
&Firefox SpiderMonkey~\cite{spiderMonkey} & $2,908,550$\\
\rowcolor{Gray}& Ubuntu 19.10 w/o kernel & \red{$600,712$}\\
\rowcolor{Gray}& Arch Linux w/o kernel& \red{$71,188$}\\
\rowcolor{Gray}\multirow{-3}{*}{OS}&Linux Kernel & \red{$36,680,915$}\\
\multirow{4}{*}{\textbf{\device}}&HDMI interceptor + overlay & $1,911$\\ 
&USB stack & $893$\\
&Crypto stack & $3,500$\\ 
&RPi tiny core Linux & \red{$121,899$} \\\hline
\end{tabular} 
\caption[\name code-base comparison]{\textbf{\name code-base comparison} with respect to some of the open-source browsers, JS engines and OSs.}
%\spacesaveL
\label{tab:loc}
\end{table}


\subsection{Codebase comparison} In Table~\ref{tab:loc}, we provide the code base and executable binary sizes of \device with respect to some of the most popular open-source browsers, JavaScript interpreter engines and OS's. All of the codes are measured with the \texttt{cloc} open-source code line counting tool. The table shows that \name has a significantly lower code base, resulting in a smaller attack surface.

%\red{\myparagraph{Effectiveness of lightbox} Lightbox is an existing technique that is evaluated in the literature. Huang et al.~\cite{huang2012clickjacking} provide a comprehensive user study of several user attention focusing mechanisms. The lightbox is effective $98\%$ of the time, while the freezing is effective in $97\%$ time. For more details of the user study, refer to Table 2 in~\cite{huang2012clickjacking}. We assume that a similar result should be expected in \name due to the similarity of the application space (web applications).}

\subsection{\device cost} We estimate that our \device prototype costs around 140 USD (Rpi4 = \$35 + HDMI-CSI =\$30 + Due = \$35 + Zero = \$40). An integrated, mass-produced device would be, of course, significantly cheaper.

