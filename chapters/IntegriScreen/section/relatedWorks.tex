\section{Related Work} 
\label{integriscreen:sec:relatedWork}

Previous work on the trusted path either relies on a trusted hypervisor that supervises a compromised virtual machine or on the use of another trusted device that serves as a second factor.

\myparagraph{Trusted hypervisors}
Trusted hypervisors and secure micro-kernels are a possible choice for the trusted path. Sel4~\cite{klein2009sel4} is a functional hypervisor that is formally verified and has a kernel size of only $8400$ lines of code. In work done by Zhou et al.~\cite{x86}, the authors proposed a generic trusted path on $x86$ systems in pure hypervisor-based design. Examples of other hypervisor-based works can be found in systems such as Overshadow~\cite{Overshadow}, Virtual ghost~\cite{criswell2014virtual}, Inktag~\cite{hofmann2013inktag}, TrustVisor~\cite{mccune2010trustvisor}, Splitting interfaces~\cite{ta2006splitting}, $SP^3$~\cite{yang2008using}.

Our approach is most similar to Gyrus~\cite{gyrus}, a system that enforces the integrity of user-generated network traffic of protected applications by comparing it with the text values displayed on the screen by the untrusted VM. However, Gyrus requires application-specific logic and does not prevent potential UI manipulation attacks.

Not-A-Bot (NAB)~\cite{nab} ensures that the data received from the client was indeed generated by the user rather than malware by having the server require a proof (generated by a trusted \emph{attester} application) of the user's keyboard or mouse activity shortly before each request.
Similarly, BINDER~\cite{binder} focuses on detecting malware break-ins and preventing data exfiltration by implementing a set of rules that correlate user input with outbound connections. While these approaches are similar to \sysname in ensuring that outgoing requests match the user's activity on the client device, our solution differs in that it allows for a fully compromised client.


\myparagraph{Trusted devices}
The assumption of a fully compromised client mandates an additional trusted device is used to secure the interaction with the remote server.
Weigold et al. propose ZTIC~\cite{weigold2011}, a device with simple user input and display capabilities, on which users confirm summary details for a banking transaction. Another approach is taken by Kiljan et al.~\cite{6978928}, who propose a simple \emph{Trusted Entry Pad}, that computes signatures of user-input sensitive values and sends them independently to the server for verification. However, such approaches either require the user to input data to an external device, which breaks the normal workflow and duplicates efforts or require the user to confirm transaction details. This leads to habituation and decreases security. The approach of continuous visual supervision improves on previous work by neither requiring additional input nor relying on user attentiveness during transaction confirmation.