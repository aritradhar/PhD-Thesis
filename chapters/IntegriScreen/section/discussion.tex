
\section{Discussion} 
\label{integriscreen:sec:discussion}

% We now discuss open questions, limitations, and possible extensions.

\myparagraph{Detecting user attention and non-repudiation}
In this chapter, the system uses hand movement to detect user's presence and activity.
However, when the mobile device is placed between the host and the user, its front facing camera is well positioned to capture the user's face.
Given the face tracking capabilities available in recent iOS and Android mobile phones, as well as recent advances in mobile camera-based eye tracking~\cite{krafka2016eye}, the system could be extended to precisely track user's attention on the screen and require that gaze is present for certain data modification.
Furthermore, if the mobile device used face recognition to continuously authenticate the user, the system could also provide non-repudiation guarantees.


\myparagraph{Non-textual UI elements}
As the first of the proposed \textit{visual supervision} paradigm, in this chapter, we focused on textual input.
However, our approach can be extended to support non-textual UI elements -- as long as their final state is shown on the screen -- such as checkboxes, sliders, or calendar widgets.
Furthermore, while we focused on text extraction, this step could be implemented by a more literal comparison of the host's screen with a screenshot of the web form, as rendered by the server with same aspect ratio and element position.

\myparagraph{Privacy and security of visual supervision}
While continuously recording one's interaction with another electronic device seems intrusive, we note that all processing in \sysname happens on the mobile device.
Therefore, the server only receives a duplicate of the data from the host.
If sending a duplicate of the data is not suitable for any reason, it is straightforward to modify \name to compute and only send a digest (similar to a TAN) to the server.

However, a visual channel gives users clear control over which data the mobile device observes, i.e., only what is shown on the screen at a given moment -- while most OS level applications typically get unrestricted access to the whole system and could violate the users' privacy in the background while keeping them oblivious.

Finally, using only a visual channel between the host and the mobile device reduces the likelihood of a smartphone compromise, since it never directly communicates with the already compromised host.