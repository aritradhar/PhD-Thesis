\section{Conclusion} 
\label{integriscreen:sec:conclusion}

In this chapter, we explore the idea of \emph{visually supervising} user's IO data to provide IO integrity against a compromised host. We use a smartphone to capture the client's screen and enforce the integrity of the web rendered on the screen, alongside the input data the user submits on the web form. We show the feasibility of this approach by developing a fully functional prototype on an Android smartphone, evaluating it with a series of experimental tests, and running a user study to measure participants' responses to simulated attacks.
Considering the rapid increase in processing power and camera quality of smartphones, but also novel platforms such as augmented reality headsets and smart home assistants, we envision such systems that supervise user's IO will be ubiquitous.

Even though \integriscreen addresses the problem of input and output integrity and makes significant progress over \integriscreen, \integriscreen is still not a completely secure and usable system. Usage of smartphones significantly increases the TCB of the system. Moreover, \integriscreen only considers keyboard input which opens up new attack surfaces. Despite \integrikey and \integriscreen provides a significant advantage over their predecessors, both follow the similar system-building approach of the existing trusted path proposals. Such a mere system-building approach does not solve the fundamental problem of trusted path. Hence, we conclude this first part of this thesis by identifying \emph{how not to build a trusted path}. 