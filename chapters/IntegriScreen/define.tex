\renewcommand*\circled[1]{\tikz[baseline=(char.base)]{
            \node[shape=circle,draw,inner sep=0.3pt] (char) {$#1$};}}

\newcommand{\app}{smartphone app\xspace}

\newcommand{\client}{host\xspace}
\newcommand{\POI}{proof-of-intent\xspace}
\newcommand{\PsOI}{proofs-of-intent\xspace}

\newcommand{\myparagraphnodot}[1]{\vspace{4pt} \noindent {\bfseries #1}\xspace}
\newcommand{\attacker}{$\mathcal{A}$\xspace}

\newcommand{\smartphone}{smartphone\xspace}
\newcommand{\md}{smartphone\xspace}

\newcommand{\atk}[1]{$\mathbf{U_#1}$\xspace}
\newcommand{\A}[1]{$\mathbf{A_#1}$\xspace}
\newcommand{\B}[1]{$\mathbf{B_#1}$\xspace}
\newcommand{\R}[1]{$\mathbf{R_#1}$\xspace}

% This is a marker to make sure we don't leave some of the values wrong while we are writing them
\definecolor{mygreen}{rgb}{0.22, 0.72, 0.21}
%\newcommand{\updatelater}[1]{\textcolor{mygreen}{#1}\xspace}
\newcommand{\updatelater}[1]{#1\xspace}

\newcommand{\MITB}{\emph{Man-in-the-browser}\xspace}

\newcommand{\numforms}{\updatelater{100}\xspace}

\newcommand{\sysname}{\textsc{IntegriScreen}\xspace}
\renewcommand{\name}{\sysname}

