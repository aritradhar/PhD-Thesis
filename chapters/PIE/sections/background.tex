\section{Background}
\label{sec:background}

\myparagraph{Keystone}
Keystone~\cite{keystone} is a TEE framework based on RISC-V that utilizes physical memory protection or PMP to provide isolation. PMP is part of the RISC-V privilege standard~\cite{riscv2019privspec} and it allows to specify access policies that can individually allow or deny reading, writing, and executing for a certain memory range. E.g., PMP can be used to restrict the operating system (OS) from accessing the memory of the bootloader. Every access request to a prohibited range gets trapped precisely in the core and results in a hardware exception. Keystone relies on a low-level firmware with the highest privilege, called security monitor (SM), to manage the PMP. 

The SM maintains its own memory separate from the OS and protected by a PMP entry. It also facilitates all enclave calls, e.g., it creates, runs, and destroys enclaves. The SM configures the PMP so that the OS can no longer access the enclave's private memory. Upon a context switch, the SM re-configures the PMP to allow or block access to the enclave. E.g., during a context switch from an enclave to the OS, the SM changes the PMP configuration such that access to the enclave memory is prohibited. Conversely, on a context switch back to the enclave, the PMP gets reconfigured to allow accesses to enclave memory. 
Because the SM is critical for the security of any enclave and the whole system, it aims to be very minimal and lean. As such, the SM is orders of magnitudes smaller than hypervisors and operating systems (15k LoC vs millions LoC~\cite{torvalds2020linux,barham2003xen}). There are also efforts to create formal proofs for such a SM~\cite{lebedev2019sanctorum}. Keystone also provides extensions for cache side-channel protections using page coloring or dynamic enclave memory. 

\myparagraph{Device Tree}
The device tree is a list that accurately describes the physical memory mappings of a platform. It describes the central processor, i.e., its speed, its ISA, and at what address its cache starts. It also includes the DRAM base address and various other components on the die, such as various internal and external buses. It is usually used by the bootloader and the OS to bootstrap the system. As some peripherals cannot be detected automatically, they must be present in the device tree, as otherwise they will not get recognized by the OS. The device tree is usually burnt into ROM and available to the bootloader and the OS. It can therefore be considered trusted.

% \subsection{Hardware Background}
% \subsubsection{RISC-V}
% RISC-V is a modern open-source instruction set architecture (ISA). In recent years, RISC-V has sparked immense interest by industry and academia, with many software and hardware proposals. As a consequence, multiple open source cores that implement various subsets of the RISC-V standard have emerged~\cite{ariane,asanovic2016rocket,riscy,asanovic2015boom}. Software-wise, RISC-V is supported by many projects, e.g., by GCC, linux, and QEMU amongst others. RISC-V has also become a popular prototype platform for security research~\cite{weiser2019timber,costan2016sanctum,keystone}.

% \subsubsection{Physical Memory Protection}
% Physical memory protection (PMP) is part of the RISC-V privilege standard~\cite{riscv2019privspec}. It allows to specify access policies based on a physical memory range. The access policies can individually allow or deny reading, writing, and executing for a certain memory range. For example, PMP can be used to restrict the operating system (OS) from accessing the memory of the bootloader. Every access request into a prohibited range gets trapped precisely in the core and results in a hardware exception. There are a fixed number of available PMP entries, typically 8 or 16~\cite{riscv2019privspec}, each of which allows to specify a separate memory range. Since PMP is part of the standard, it is already included in many currently available cores~\cite{riscy,asanovic2016rocket}.



% \subsection{Trusted Execution Environments (TEE)}

% Traditional trusted execution environments (TEE) introduce a small protected program called enclave. An enclave runs on the processor isolated from the attacker-controlled OS and hypervisor. Enclaves provide code integrity and data confidentiality, even in the presence of a physical adversary. Integrity of the enclave code at the time of deployment is ensured by remote attestation while the enclave data confidentiality and integrity during runtime are provided by various hardware mechanisms. For example, Intel SGX uses memory encryption and the Memory Management Unit (MMU).
%, whereas RISC-V Keystone~\cite{keystone} uses PMP to isolate the enclave data from the untrusted OS and other applications. 

%\subsection{Keystone}

