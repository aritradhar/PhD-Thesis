
\section{Data Confidentiality in a Local Physical Attacker Setting}
\label{pie:sec:localAttacker}


A physical local attacker is what is traditionally considered when developing TEEs~\cite{costan2016intel,keystone}. With such an attacker only the processor package is trusted, alongside with some low-level software, called security monitor in Keystone and $\mu$Code on Intel SGX. This software is crucial as it incorporates the TEE functionality into RISC-V or Intel processors respectively. In addition, we also trust the peripheral package since a rogue peripheral obviously breaks any secrecy and isolation requirement. We assume the user to be cooperating and interested in getting a correct measurement from the peripheral. So active conspicuous attacks are out of scope.  If, however, the physical world is completely under control of the adversary, then the adversary can change some properties of the physical world such as the temperature or GPS data. Obviously this will lead to a maliciously modified sensor reading and can result in malfunction. However, this attack vector is considered out-of-scope because it usually presents a major hurdle for primitive adversaries (i.e., users). Nevertheless, a sophisticated adversary can take advantage of this and cause tampered sensor readings. Note that this only concerns measurements of the physical world and external accelerators are not affected by this. By trusting the package, we assume that the local attacker can not physically tamper with the processor or the peripherals. However, all connections between the processor and peripherals are fair game and can be fully controlled by an adversary. E.g., on a commercially available desktop platform, the processor package and peripheral ICs (USB, PCI Express) are trusted. However, we also assume that the attacker can also plug her own compromised peripherals into the platform. 



  \begin{figure}[tbp]
   \centering
   \includegraphics[trim={0 8.5cm 23cm 0}, clip, width=0.5\linewidth]{chapters/PIE/images/memoryEnc.pdf}
   \caption[Different cryptographic engines in \name]{Different cryptographic engines in relation to different components of \name. AE, CE and MEE stands for application enclave, controller enclave and memory encryption engine respectively.}
   \label{fig:encEngine}
  \end{figure}

\subsection{Cryptographic Engines and Key Management}

  \name relies on memory encryption engine (MEE) and bus encryption engine (BEE) to protect from a local physical attacker. Figure~\ref{fig:encEngine} shows these cryptographic engine in context to various \name programming model components.

 \myparagraph{Memory encryption engine (MEE)} All the enclaves that are running on the CPU cores (\app and \ce) and all the shared memory regions are encrypted by the MEE. Due to this, the attacker who has physical access to the memory module can not observe or manipulate the shared memory contains. As the only the CPU cores access these memories, only one key is sufficient to encrypt and preserve the integrity of the data in all the memory regions. Note that the CPU keeps the integrity tree of all of the shared memory and enclave memory regions.

 \myparagraph{Bus encryption engine (BEE)} MEE is not sufficient in the communication between the CPU core enclaves and peripheral enclaves. In MEE, as mentioned before, the CPU keeps the integrity tree. This is sufficient as only the CPU cores write and read to and from the DRAM. However, when we consider the communication between the CPU core enclaves and the peripheral enclaves, the integrity tree only at the CPU core does not ensure the integrity of the data coming from the peripherals. Due to this, we leverage the bus encryption engine that is present both in the CPU and the peripherals. During the initial attestation process, the CPU and the peripherals derive the session key that is used as the BEE keys. Note that every peripheral derives its own key to communicate with the CPU. This was, an attacker-controlled peripheral can not observe or modify the data on the bus coming from another peripheral. 


 \begin{figure}[t]
   \centering
   \includegraphics[trim={0 13cm 18cm 0}, clip, width=0.7\linewidth]{chapters/PIE/images/peripheral.pdf}
   \caption{An example peripheral with modifications for \name{} that adds the bus encryption engine (BEE) and a keystorage to communicate with the CPU.}
   \label{fig:sensor}
 \end{figure}



 \subsection{Modified Peripherals}


 The peripherals themselves require some changes in order to be part of a \nameenclave. Here we describe one potential peripheral configuration and what modifications are necessary. The modifications to the peripherals enhance their functionality by remote attestation, and bus encryption for the communication However, the complexity of the peripherals and their modifications might change depending on the use case. An example peripheral could be an analog sensor (e.g., a thermometer) that contains a small digital signal processor (DSP) that convert the analog reading to digital signal.  Such devices usually come with a limited memory to store the configuration parameters/ look-up tables or some past sensor readings. the only required hardware changes are the necessary secure storage for keys and certificates provisioned by the manufacturer. All cryptographic algorithms can be performed by the microcontroller since the performance requirements are rather low. An example of such modified peripheral with a microcontroller, BEE and key-storage is depicted in Figure~\ref{fig:sensor}.