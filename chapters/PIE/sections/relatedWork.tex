\section{Related Work}
\label{sec:relatedWork}

% In this section, we discuss some of the most relevant works to \name. We also discuss the main differences with \name to these works.

\setcounter{para}{0}
\paragraph{TEE-based solutions} There exist a number of solutions of integrating external devices to widely-deployed TEEs. SGXIO~\cite{weiser2017sgxio} aims to allow Intel SGX enclaves to interact with input-output devices under the remote adversary model. SGXIO uses a trusted hypervisor which virtualizes peripherals. SGXIO is static, i.e., all the peripherals have to be set up at boot time and no changes are allowed during runtime (connect new peripherals, etc.). It is not clear how enclaves are created and get access to a peripheral while preserving the confidentiality of previous enclaves that used said peripheral. 
%Essentially, SGXIO allows for a static extension of the hardware TCB.

Graviton~\cite{volos2018graviton} is a TEE that enables isolated concurrent enclaves on a graphics card. Graviton would fit very well within a \name{} as it is an excellent example of an enclave on \sphw and it shows that even some of the most powerful accelerators can be extended with a local TEE. Visor~\cite{visor} is based upon Graviton~\cite{volos2018graviton} and proposes a hybrid TEE that spans over both CPU and GPU. These enclaves communicate between themselves securely. Visor is aimed towards privacy-preserving video analytics where the computation pipeline is shared between the CPU (non-CNN workloads) and the GPU (CNN workloads) to increase efficiency. 

HETEE~\cite{zhu2020hetee} is another proposal to extend TEEs to accelerators (specifically GPUs) without requiring changes to existing CPUs/GPUs. HETEE focuses on datacenter applications and proposes an extra hardware box per rack that is protected from physical attacks. This box also contains all accelerators which it then connects to compute servers in the same rack. Each enclave then runs on a dedicated compute server and a connected accelerator. In essence, the HETEE box provides secure routing of accelerators to dedicated compute servers. In contrast to HETEE, we aim to be able to execute multiple composite enclaves on the same compute server. 


ARM TrustZone is a system TEE provided by ARM for their system-on-chips (SoC)~\cite{winter2008trusted}. TrustZone applications run on top of a secure OS that is trusted and isolated from the standard OS (also known as the rich OS). 
%Isolation between the two worlds is achieved by an extra bit on the bus. However, there is no isolation between different TrustZone applications. Due to this limitation, mobile phone manufacturers usually only allow TrustZone applications that are signed by them.
TrustZone only provides the lower level isolation property between the rich OS and the secure OS with an extra bit on the bus. Everything else, i.e., isolation between TrustZone applications, remote attestation, etc., has to be added to the secure OS~\cite{ning2014samsungknox}. Due to this limitation, mobile phone manufacturers usually only allow TrustZone applications that are signed by them. 
% Multiple existing works~\cite{brasser2019sanctuary,hua2017vtz} improve on the capabilities of TrustZone. 
Sanctuary~\cite{brasser2019sanctuary} extends TrustZone with user-space enclaves. Sanctuary achieves isolation by running enclaves in their own address space in the normal world. However, Sanctuary does not extend to external \sphw.

Some of the proposals~\cite{TruZ-Droid,trustUI,SeCloak,VButton} enable additional security properties such as a trusted path by enabling direct pairing of peripherals (e.g., the touchscreen) to the TrustZone application. However, these are only geared towards IO operation for trusted path and cannot be dynamically extend to support generic devices.

\paragraph{Other isolation methods} Minimal hypervisors or operating systems~\cite{herder2006minix,klein2009sel4} can also achieve isolation, and some are even formally verified~\cite{klein2009sel4}. Usually, such hypervisors do not include attestation, but the cost of adding that should be low. A \name{} could also be based on a microkernel such as seL4. One would have to add an interface for the malicious OS running in a virtual machine to interact with enclaves that run directly on top of seL4, similar to other pure hypervisor-based isolation systems~\cite{virtualGhost,Overshadow,InkTag,TrustVisor,SplittingInterfaces,terra}. It might even be possible to formally prove such modifications to provide an even stronger assurance of isolation. 
% Similar to other pure hypervisor-based isolation system include Virtual Ghost~\cite{virtualGhost}, Overshadow~\cite{Overshadow}, InkTag~\cite{InkTag}, TrustVisor~\cite{TrustVisor}, Splitting Interface~\cite{SplittingInterfaces}, Terra~\cite{terra} etc. 
The hypervisor is also in charge of the scheduling, resulting in a significantly bigger TCB than \name. Moreover, in none of the hypervisor-based proposals, platform awareness, and platform-wide attestation are considered. Isolation is the sole objective of these proposals.

\paragraph{Bump in the wire-based solutions} Fidelius~\cite{Fidelius}, ProtectIOn~\cite{protection}, IntegriScreen~\cite{integriscreen}, FPGA-based overlays~\cite{fpga_overlay}, IntegriKey~\cite{integrikey} are some of the trusted path solutions that use external trusted hardware devices as intermediaries between the platform and IO devices. These external devices create a trusted path between a remote user and the peripheral and enable the user to exchange sensitive data securely with the peripheral in the presence of an attacker-controlled OS. Such solutions provide a loose notion of platform awareness and are focused on IO devices. Platform-wide attestation and strong isolation guarantee are out-of-scope of such proposals. 

\setcounter{para}{0}


% \begin{table*}[t]
%   \centering
%   \resizebox{\textwidth}{!}{%
%     
\newcommand{\yes}{\CIRCLE}
\newcommand{\no}{\Circle}
\newcommand{\yesno}{\LEFTcircle}
% \newcolumntype{Y}{>{\centering\arraybackslash}X}
\begin{tabular}{@{}lcccccccccccccccc@{}}
% \begin{tabularx}{\textwidth}{@{}lcclclYYYlYYYlYYY@{}}
% \footnotesize
% \begin{tabular}{@{}lcccccccccccccccc@{}}
\toprule
    & \multicolumn{4}{c}{Platform Awareness} &  & \multicolumn{11}{c}{Isolation between Partitions} \\ \cmidrule(lr){2-5} \cmidrule(l){7-17} 
    & \multicolumn{2}{c}{Attestation} &  & \multirow[c]{2}{*}[-3pt]{\begin{tabular}[c]{@{}l@{}}Bidirectional\\ Awareness\end{tabular}} &  & \multicolumn{3}{c}{Spatial} &  & \multicolumn{3}{c}{Temporal} &  & \multicolumn{3}{c}{Fault} \\ \cmidrule(lr){2-3} \cmidrule(lr){7-9} \cmidrule(lr){11-13} \cmidrule(l){15-17} 
    & Enclave & Platform &  &  &  & \multicolumn{3}{c}{Core/SoC/Platform} &  & \multicolumn{3}{c}{Core/SoC/Platform} &  & \multicolumn{3}{c}{Core/SoC/Platform} \\ \midrule
SGX~\cite{costan2016intel} & \yes & \no &  & \no &  & \yes & \no & \no &  & \no & \no & \no &  & \yes & \yes & \no \\
Sanctum~\cite{costan2016sanctum} & \yes & \no &  & \no &  & \yes & \no & \no &  & \yes & \yes & \no &  & \yes & \yes & \no \\
Keystone~\cite{keystone} & \yes & \no &  & \no &  & \yes & \no & \no &  & \yesno & \yesno & \no &  & \yes & \yes & \no \\
Sancus~\cite{noorman2013sancus} & \yes & \no &  & \no &  & \yes & ? & \no &  & \no & \no & \no &  & \yes & ? & \no \\
TrustZone & ? & \no &  & \no &  & \yesno & \yesno & \no &  & \no & \no & \no &  & \no & \no & \no \\
Samsung Knox & \yes & \no &  & \no &  & \yesno & \yesno & \no &  & \no & \no & \no &  & \no & \no & \no \\
Sanctuary~\cite{brasser2019sanctuary} & \yes & \no &  & \no &  & \yes & \no & \no &  & \no & \no & \no &  & \yes & ? & \no \\
TPM & ? & \no &  & \no &  & \no & \no & \no &  & \no & \no & \no &  & \no & \no & \no \\
DRTM & \yes & \no &  & \no &  & \no & \no & \no &  & \no & \no & \no &  & \no & \no & \no \\
XOM~\cite{lie2000xom} & \yes & \no &  & \no &  & \no & \no & \no &  & \no & \no & \no &  & \no & \no & \no \\
Aegis~\cite{suh2003aegis} & \yes & \no &  & \no &  & \yes & \no & \no &  & ? & ? & \no &  & ? & ? & \no \\
% Bastion & \yes & \no &  & \no &  & \yes & \no & \no &  & ? & ? & \no &  & ? & ? & \no \\ \bottomrule
% \end{tabularx}
\end{tabular}
%   }
%   \caption{Comparison between traditional TEEs.\todo{Fill remaining question marks. Add daggers for all isolation facilitated by MMU}\todo{Change to the new requirements}}
%   \label{tab:comparison}
% \end{table*}

\iffalse
\setcounter{para}{0}
\mypara{SGXIO} SGXIO is a proposal by Weiser et al.~\cite{weiser2017sgxio} that builds on top of Intel SGX with the goal of allowing SGX to interact with input-output devices. They achieve that by introducing a trusted hypervisor that allows enclaves to access virtualized peripherals. Similar to our approach, SGXIO also considers a remote adversary. However, SGXIO is rather static in nature, i.e., all the peripherals have to be set up at boot time. After the system setup phase, no changes are allowed (connect new peripherals, etc.). It is not clear how enclaves are created and get access to a peripheral while preserving the confidentiality of previous enclaves that used said peripheral. Essentially, SGXIO allows for a static extension of the hardware TCB.

\mypara{Graviton and systems based on it} Graviton~\cite{volos2018graviton} is a TEE that runs on an accelerator such as a graphics card. It can provide isolation between the data of multiple stakeholders that run tasks on the GPU concurrently. It also provides remote attestation of an enclave on the accelerator. Graviton was evaluated on a modern graphics card and shows that the predominant overhead stems from encryption of the communication to the processor. However, they also demonstrate that the overhead of around 20\% is tolerable. Graviton would fit very well within a \name{} as it is an excellent example of an enclave on a peripheral. It provides isolation for secret data and attestation reports. In addition, it shows that even some of the most powerful accelerators can be extended with a local TEE. Visor~\cite{visor} is a system built upon Graviton~\cite{volos2018graviton} that proposes a hybrid TEE that spans over both CPU and GPU. Visor is aimed towards privacy-preserving video analytics where the computation pipeline is shared between the CPU (non-CNN workloads) and the GPU (CNN workloads) to increase efficiency. Visor addresses micro-architectural based side-channel attacks where a local physical attacker can use the data-dependent memory access patterns (e.g., branch-prediction, cache-timing, or controlled page fault attacks) to reveal elements on the video analysis (e.g., leaks pixel patterns). The communication between the CPU and GPU enclaves are encrypted. Additionally, Visor ensures that the traffic pattern between the CPU and GPU enclaves is independent of the video content.

\mypara{ARM TrustZone and systems based on it} TrustZone is a system TEE provided by ARM for their system-on-chips (SoC)~\cite{winter2008trusted}. TrustZone applications run on top of a secure OS that is trusted and isolated from the standard operating system (also known as the rich OS). Isolation between the two worlds is achieved by an extra bit on the bus. 
%and some additional configurable components such as the TrustZone Address Space Controller (AZASC). The AZASC is a dedicated hardware component on the SoC that may permit and refuse some access to memory. 
However, there is no isolation between different TrustZone applications. Due to this limitation, mobile phone manufacturers usually only allow TrustZone applications that are signed by them. % --  a restricted version of the local attestation. 
TrustZone only provides the lower level isolation property between the rich OS and the secure OS. Everything else, i.e., isolation between TrustZone applications, remote attestation, etc., has to be added to the secure OS~\cite{ning2014samsungknox}. There have been many proposals that try to improve on the capabilities of TrustZone~\cite{brasser2019sanctuary,hua2017vtz}. Sanctuary~\cite{brasser2019sanctuary} enables user-space TrustZone enclaves. Sanctuary achieves isolation by running enclaves in their own address space in the normal world. However, Sanctuary is very similar to Intel SGX, and thus, it does not extend to peripherals.
%Similar to these proposals, TrustZone could also be extended in the direction of \name{} to support \nameenclave{}s. 
There exists proposals that enable additional security properties such as a trusted path by enabling direct pairing of peripherals (e.g., the touchscreen) to the TrustZone application. Such proposals include TruZ-Droid~\cite{TruZ-Droid}, TrustUI~\cite{trustUI}, SeCloak~\cite{SeCloak}, VButton~\cite{VButton}. All of these solutions do not provide any form of peripheral attestation similar to \name{}. They also do not consider other peripherals or how to dynamically allow an enclave to access a peripheral previously used by another enclave. Moreover, as mentioned earlier, the absence of isolation between the TrustZone applications makes the proposal mentioned above weak in inter enclave isolation guarantee.

\subsection{Other Isolation Methods}

Minimal hypervisors or operating systems~\cite{herder2006minix,klein2009sel4} can also achieve isolation, and some are even formally verified~\cite{klein2009sel4}. Usually, such hypervisors do not include any attestation primitives, but the cost of adding them is low. A \name{} could also be based on a microkernel such as seL4. One would have to provide an interface for the malicious OS running in a virtual machine to interact with enclaves that run directly on top of seL4. It might even be possible to formally prove such modifications to provide an even stronger assurance of isolation. Other pure hypervisor-based isolation system include Virtual Ghost~\cite{virtualGhost}, Overshadow~\cite{Overshadow}, InkTag~\cite{InkTag}, TrustVisor~\cite{TrustVisor}, Splitting Interface~\cite{SplittingInterfaces}, Terra~\cite{terra} etc. In all of these proposals, the hypervisors also take care of the scheduling. Hence the TCB is a lot bigger than our approach. Moreover, in none of the hypervisor-based proposals, peripheral awareness, and platform-wide attestation are considered. Isolation is the sole objective of these works.
\fi